\section{Prerušenia vo vstavaných systémoch}
\noindent Vo vstavaných systémoch prerušenie upozorní procesor na udalosť, ktorá nastala a je potrebné aby jej procesor venoval pozornosť. Často ide o signály vygenerované perifériami procesora. Procesor na prerušenie zareaguje pozastavením vykonávania aktuálnej aktivity, uložením jej stavu a vykonaním funkcie nazývanej aj obsluha prerušenia. V anglickej literatúre sa používa výraz interrupt service routine \begin{math}(ISR)\end{math}, preto aj my budeme ďalej používať skratku  \begin{math}ISR\end{math}. Po vykonaní  \begin{math}ISR\end{math} sa procesor vráti k predtým vykonávanej aktivite. \par
Prerušenia je možné vnímať aj ako udalosti. Pri ich využívaní v programe je preto možné hovoriť o programovaní pomocou udalosťami riadenej paradigmy. Takýto program je možné označiť ako \begin{math}P=Task || ISR\end{math}, kde \begin{math}Task\end{math} je hlavný program pozostávajúci z jednej alebo viacerých úloh (vlákien) a  \begin{math}ISR\end{math} kde \begin{math}ISR=ISR_1||ISR_2||ISR_3||..||ISR_N\end{math} sú obsluhy prerušení.
Indexy daných \begin{math}ISR\end{math} predstavujú číslo prerušenia. Čím je číslo väčšie, tým menšiu prioritu dané prerušenie má \cite{wangAutomaticDetectionValidation2017}. Za plánovanie podľa priorít ISR zodpovedá
hardvérový mechanizmus. Zatiaľčo ostatné úlohy (Task) sú plánované kernelom podľa ich softvérových priorít. Hardvérové priority sú nad softvérovymi prioritami \cite{leyva-del-foyoCustomInterruptManagement}. \par Prerušenia sú väčšinovo využívané na indikovanie dvoch typov udalostí. Prvou je udalosť vykonaná používateľom, kde patrí napríklad stlačenie tlačidla. Druhou je prijatie dát zo špecifického zariadenia. Ako príklad môžeme uviesť UART protokol .