\section{Analýza problému}
V súčasnej dobe patrí platforma Arduino medzi populárne elektronické platformy na svete. Jeho výhodou je otvorený zdrojový kód, jednoduchosť použitia, ale aj možnosť vytvárať komplexnejšie projekty. Vďaka týmto spomenutým aspektom je Arduino využívané úplnými začiatočníkmi, ale aj prokročilejšími používateľmi \cite{WhatArduinoArduino}.
 Práve pri komplexnejších projektoch sa často môžeme stretnúť časovo kritickými 
 úlohami, ako aj s nutnosťou zastaviť aktuálne vykonávaný kód a reagovať na externý vstup. To pri štandardnom sekvenčnom písaní kódu nemusí byť jednoduché a často to môže končiť kódom, ktorý je len veľmi ťažko udržiavateľný. Preto sa na to pri tomto type programovania využíva udalosťami riadená paradigma. Táto paradigma môže byť ďalej modelovaná napríklad pomocou Petriho sieti \cite{bastidePetriNetBased1995}, alebo aj za pomoci konečných stavových automatov \cite{dashEventDrivenProgramming2011}.
 TODO: DOKONCIT 