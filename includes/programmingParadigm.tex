\section{Programovacie paradigmy}
\noindent Programovacia paradigma je spôsob riešenia programátorských problémov \cite{Samuel2018AnII}. Poznáme veľké množstvo programovacích paradígm. Medzi ne patria aj nasledujúce štyri:
\begin{itemize}
  \item Imperatívna paradigma.
  \item Funkcionálna paradigma.
  \item Objektovo orientovaná paradigma.
  \item Udalosťami riadená paradigma.
\end{itemize}

\subsection{Imperatívna paradigma}
\noindent Základnou myšlienkou je príkaz, ktorý má merateľný dopad na aktuálny stav programu. Pri imperatívnej paradigme je rovnako dôležité aj poradie príkazov. Stav programu po vykonaní príkazov v nejakom poradí nemusí byť rovnaký ak by boli príkazy vykonané v inom poradí \cite{imperativParadigm}. 

\subsection{Funkcionálna paradigma}
\noindent Je postavená na koncepte matematických funkcií, ktoré používajú podmienené výrazy a rekurziu na vykonanie výpočtu. Zameriava sa na výsledok, nie na spôsobom akým sa k výsledku dopracovať. Jednou z jeho základných vlastností je nemennosť dát. To znamená, že ak už bola nejaká štruktúra vytvorená nieje možné meniť jej vnútorný stav. V prípade potreby zmeny stavu je nutné celú štruktúru vytvoriť odznova. Medzi najpoužívanejšie funkcionálne jazyky patrí Haskell, Clojure alebo SQL \cite{functionalParadigm}.

\subsection{Objektovo orientovaná paradigma}
\noindent Paradigma, ktorej myšlienka je inšpirovaná svetom okolo nás. Čokoľvek v môže byť objekt. Auto, dom, kniha či dokument. Každý z týchto objektov má nejaké vlastnosti a nejaké správanie. Rovnakým spôsobom aj modelujeme program. Takéto rozmýšľanie o objektoch nám umožňuje písať kód, ktorý je jednoduchší na pochopenie a zároveň je vďaka tomu možné kód rozdeliť na menšie časti. Tieto časti kódu sú následne jednoduchšie udržiavateľné. Medzi hlavné princípy objektovo orientovaného prístupu patrí dedičnosť, zapuzdrenosť, polymorfizmus a abstrakcia \cite{objectOrientedParadigm}. Populárne programovacie jazyky podporujúce túto paradigmu sú Python, C\# a Java \cite{stack-overflow-survey-2020}.

\subsection{Udalosťami riadená paradigma}
\noindent V tejto paradigme je tok programu určený udalosťami. Udalosti môžu byť rôzne. Často sa využíva pri programovaní aplikácii s užívateľským rozhraním, ktoré musia reagovať na vstupy z periférii počítača (napríklad kliknutie na tlačidlo, písanie textu). Program pozostáva z hlavnej slučky programu a obslužných funkcií dostupných udalostí. Úlohou hlavnej slučky je volanie obslužných funkcií na základe typu udalosti. Hlavná slučka programu môže byť taktiež nahradená pomocou hardvérových prerušení.