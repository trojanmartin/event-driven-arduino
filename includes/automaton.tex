
\subsection{Automaton}
\noindent Automat je názov rosiahlej open-source Arduino knižnice, ktorej hlavným cieľom
je modelovanie aplikácii pomocou konečných automatov. Vznikla v roku 2015 a je dostupná pod MIT licenciou.
\par Knižnica umožňuje vytváranie niekoľkých súbežne spustených konečných automatov, ktoré medzi sebou spoluprácujú. Jednotlivé komponenty sa dokážu navzájom spúštať a vytvárať zložité riadiace štrukúry. Automaton obsahuje veľké množstvo zabudovaných komponentov. Uživateľovi však dáva aj možnosť vytvárať si vlastné a kombinovať ich s už existujúcimi komponentami. Medzi základné stavebné kamene knižnice patria dve triedy: 
\begin{enumerate}
    \item Trieda \textit{Automaton}
\end{enumerate}

\noindent \par Táto trieda prestavuje zabudovaný plánovač. Stará sa o cyklenie konečných automatov. Obsahuje tri základné funkcie. Funckiu \textit{add()}, \textit{run()} a funckiu \textit{delay()}. Funckia \textit{run} vykoná jeden cyklus všetkých stavový strojov, ktoré trieda Automaton obsahuje. Pomocou funkcie \textit{add} je možné explicitne pridávať stavové stroje. To však väčšinou nieje nutné, pretože aplikácia zvykne obsahovať len jednu inštanciu triedy Automaton a stavové stroje sú pridané implicitne pri štarte programu. Poslednou je funkcia \textit{delay}. Tá zabezpečí cyklenie strojov pokým neuplynie príslušný počet milisekúnd.  Pri štadardnom použivaní knižnice je táto trieda použitá len vo funckii loop a to nasledoným spôsobom:

\begin{lstlisting}[
    label={lst:main-c},
    language=c
  ]
void loop() {
    automaton.run();
}

\end{lstlisting}

\begin{enumerate}[resume]
    \item Abstraktná trieda \textit{Machine}
\end{enumerate}

\noindent \par Pre vytvorenie nového stavového stroja je nutné zdediť triedu \textit{Machine}.
Tá predstavuje základ konečného automatu a definujé základné správanie a metódy, ktoré musí kažý konečný automat splniť. Medzi tieto základné metódy patrí \textit{begin()}, \textit{event()} a \textit{action()}. 
\par Funkcionalitu metód a samotnej triedy si predstavíme na jednoduchom príklade blikajúcej LED žiarovky. Blikanie namodelujeme ako konečný automat. Základom je nová trieda, ktorá dedí z už spomínanej triedy \textit{Machine}.
\begin{lstlisting}[
    label={lst:main-c},
    language=c
  ]
  #include <Automaton.h>

  class Atm_blink : public Machine {
  
    public:
      Atm_blink( void ) : Machine(){};
  
  };
\end{lstlisting}

Na to aby sme donútili LED žiarovky blikať, potrebujeme poznať pin na ktorom je 
pripojená a časovač. Časovač zabezpečí pravidelnú zmenu stavu konečného automatu pomocou vyvolania udalosti. Ďalšímy premennými budú stavy, aké môže automat nadobudnúť, spolu s udalosťami aké môžu nastať.

\begin{lstlisting}[
    label={lst:main-c},
    language=c
  ]
  #include <Automaton.h>

  class Atm_blink : public Machine { 

    short pin;
    atm_timer_millis timer;

    enum { LED_ON, LED_OFF }; // STATES
    enum { EVT_TIMER, ELSE }; // EVENTS
    enum { ENT_ON, ENT_OFF }; //ACTIONS

    public:
      Atm_blink( void ) : Machine(){};
  
  };
\end{lstlisting}
Z kódu je zrejmé, že automat môže nadobúdať dva stavy: LED\_ON, ktorý reprezentuje stav keď LED žiarovka svieti. LED\_OFF samozrejme stav, keď je žiarovka vypnutá. Medzi udalosťami sa nachádza EVT\_TIMER, ktorý reprezentuje udalosť vygenerovanú časovačom. Automat taktiež definuje dve akcie. Zapnutie a vypnutie žiarovky.
\par Nasleduje metóda begin:
\begin{lstlisting}[
    label={lst:main-c},
    language=c
  ]
Atm_blink & begin( int attached_pin, int blinkrate ) {
    const static state_t state_table[] PROGMEM = {
      /*            ON_ENTER    ON_LOOP  ON_EXIT    EVT_TIMER  ELSE */
      /* IDLE    */  ENT_OFF,        -1,      -1,          -1,   -1,
      /* LED_ON  */   ENT_ON,        -1,      -1,     LED_OFF,   -1,
      /* LED_OFF */  ENT_OFF,        -1,      -1,      LED_ON,   -1,
    };
    Machine::begin( state_table, ELSE );
    pin = attached_pin; 
    timer.set( blinkrate ); 
    pinMode( pin, OUTPUT ); 
    return *this;          
}
\end{lstlisting}

Hlavným komponentom je takzvaná stavová tabuľka. Tá reprezentuje to ako sa automat bude správať, aké prechody medzi stavmi sú definované a aké akcie sa majú vykonať. V riadkoch sa nachádzajú stavy, ktoré je možné nadobudnúť. V stĺpcoch sú udalosti, ktoré môžu nastať. Ako príklad môžeme uviesť riadok 2, ktorý reprezentuje stav svietiacej žiarovky a prvý stĺpec s názvom ON\_ENTER. Ten reprezentuje udalosť vstupu do stavu. Hodnota na súradnici 2:1 teda hovorí o  akcii, ktorá sa ma udiať, keď automat vstúpi do stavu LED\_ON. V našom prípade to je zapnutie LED žiarovky. Štvrtým stĺpcom je EVT\_TIMER. Táto udalosť nastane pri vypršaní časovača. Definujeme teda, že v prípade svietacej žiarovky a vypršania časovača chceme prejsť do stavu vypnutej žiarovky, respektíve opačne.  Mezi ďalšie kroky v metóde begin patrí nastavenie časovača a pinu.
\par Pomocou metódy event konečný automat získava informáciu o tom či nejaká udalosť nastala. V našom prípade metóda vyzerá nasledovne:
\begin{lstlisting}[
    label={lst:main-c},
    language=c
  ]

int event( int id ) {
    switch ( id ) {
        case EVT_TIMER :
            return timer.expired( this );        
        }
    return 0;
}
\end{lstlisting}
Parameter id reprezentuje udalosť o ktorej chce knižnica informáciu. V tomto prípade je to EVT\_TIMER. Ak udalosť nastala funkcia musí vratiť 1, inak 0. Vidíme, že to či udalosť nastala závisí od toho či časovač vypršal. 
\par Poslednou funkciou je action. Vyzerá podobne ako funkcia event, avšak jej úlohou je spúštanie udalostí. 
\begin{lstlisting}[
    label={lst:main-c},
    language=c
  ]

  void action( int id ) {
    switch ( id ) {
      case ENT_ON :
        digitalWrite( pin, HIGH );
        return;
      case ENT_OFF :
        digitalWrite( pin, LOW );
        return;
     }
  }
}
\end{lstlisting}
Vstup reprezentuje akciu, a na základe toho funkcia vykoná požadovanú akciu. V našom prípade zapne/vypne žiarovku.

\par Po namodelovaní blikujúcej žiarovky je následne použitie veľmi jednoduché. Celý program bude veľmi krátky a vyzerá nasledovne:
\begin{lstlisting}[
    label={lst:main-c},
    language=c
  ]

  Atm_blink led;

  void setup() {
    led.begin( 4, 200 );        // Setup a blink machine
    led.trigger( led.EVT_ON );  // Turn it on
  }
  
  void loop() {
    led.cycle();
  }
\end{lstlisting}
Ako prvé vytvoríme inštanciu konečného automatu blikajúcej LED žiarovky. Vo funkcii setup natavíme pin a časový interval blikania. Posledným krokom bude vyvolanie udalosti pre prvé zapnutie žiarovky. Takto nastavený automat potom použijeme vo funkcii loop, kde jedinou volanou funkciou bude loop. Ten zabezpečí cyklenie automatu a tým pádom aj blikanie LED žiarovky \cite{tinkerspyAutomaton2021}.
\par Automaton je veľmi mocný nástroj, pomocou ktorého je možné modelovať komplikované systémy. Dovoľuje definíciu rôznych automatov, ktoré je možné prepájať. Medzi jeho silné stránky patrí rozšíriteľnosť o nové automaty a zároveň aj veľké množstvo už existujúcich stavových strojov. Udalosti, ktoré spúšťajú akcie sú rovnako ako pri knižnici Eventually generované softvérovo. Taktiež tu teda môže prísť k veľkej latencii medzi nastaním udalosti a jej obsluhov. Zvlášť pokiaľ ide o hardvérové prerušenia.