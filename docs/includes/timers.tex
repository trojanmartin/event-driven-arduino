\setcounter{tocdepth}{4}
\setcounter{secnumdepth}{4}

\subsection{AVR časovače a počítadlá}
\noindent

Ako názov napovedá, časovače sú určené na meranie času. Často ich nazývame aj počítadlá,
pretože vo svojom základnom princípe časovače len počítajú počet dokončených periód zdrojových hodín.
Každý časovač má svoj vlastný register, kde je uložená jeho aktuálna hodnota a taktiež každý potrebuje zdroj hodín. V mikrokontroléroch často existuje niekoľko možných zdrojov
hodín a ich výber je možné určiť pomocou prepínača. Niektoré mikrokontroléry umožňujú aj použitie
externých hodín \cite{IntroductionMicrocontrollerTimers}. Pri rôznych architektúrach sa môže spôsob fungovania časovačov líšiť. V tejto diplomovej práci sa budeme venovať
mikrokontroléru s architektúrou AVR, preto aj nasledujúci opis fungovania časovačov sa vzťahuje na architektúru AVR.\par
Fungovanie časovača si najlepšie opíšeme na konkrétnom príklade.
Uvažujme časovač s 16-bitovým registrom a zdrojom hodín s frekvenciou 1 \acrshort{kHz}. Po každom uplynutí cyklu v hodinách, časovač
zvýši hodnotu uloženú v registri o 1. Dĺžku jedného cyklu časovača je možné vyjadriť vzorcom:

\begin{equation}
    \begin{aligned}
        Dĺžka\:cyklu = \frac{1}{Frekvencia\:hodín}
    \end{aligned}
\end{equation}

To znamená, že najkratší časový úsek, ktorý je možné pomocou časovača odmerať sa
rovná dĺžke periódy vstupných hodín. V našom príklade s hodinami s frekvenciou 1 \acrshort{kHz} je dĺžka periódy nasledovná:
\begin{equation}
    \begin{aligned}
        Dĺžka\:cyklu & = \frac{1}{Frekvencia\:hodín} \\
                     & = \frac{1}{1kHz}              \\
                     & = \frac{1}{ 1000 Hz}          \\
                     & =  0.001\:sekundy
    \end{aligned}
\end{equation}
Dĺžka periódy je 0.001 sekundy. Z toho vyplýva, že časovač je schopný odmerať dĺžku času, ktorý je násobkom 0.001. Ak necháme časovač napočítať do 100, potrvá to 100 krát 0.001 sekundy, čiže 0.1 sekundy \cite{cameraNewbieGuideAVR2015}.
Tu treba poznamenať, že v našom príklade počítame s frekvenciou 1kHz, čo je pomerne nízka frekvencia a dnešné mikrokontroléry sú podstatne rýchlejšie.
Ako príklad uvedieme mikrokontrolér Arduino Mega2560, ktorý používa frekvenciu 16MHz čo je rádovo viac ako 1kHz \cite{ArduinoMega2560}.
\par Pre reálnejšiu predstavu preto pokračujme v našom príklade s časovačom s frekvenciou 1MHz. Časovač má k dispozícii 16-bitový register. Maximálny počet
možných hodnôt v registri je teda $2^{16}$ (65536, maximálna hodnota je 65365).
Po dosiahnutí maximálnej hodnoty register pretečie a časovač začne počítať znova od 0.
\begin{figure}[!h]
    \centering
    \includegraphics[width=0.70\textwidth]{img/timer.png}
    \caption{Zmena hodnoty v 16 bitovom registri časovača}
    \label{figure:timer1}
\end{figure}

Pri hodinách s frekvenciou 1MHz časovač dosiahne maximálnu hodnotu registra približne každých 66 milisekúnd a dĺžka jedného cyklu je $\frac{1}{1000000}$ sekundy, čo je 1 μs \cite{cameraNewbieGuideAVR2015}.

\subsubsection{Preddelička}
\noindent \par
Pri použití 16-bitového registra je 66 milisekúnd najdlhší časový interval, ktorý dokážeme odmerať pokým register nepretečie a časovač nezačne počítať od 0.
Pre nás ľudí je to veľmi krátky časový úsek. Často chceme dosiahnuť interval, ktorý je dlhší a pre nás badateľný. Príkladom takého intervalu je 1 sekunda.
Existuje viacero spôsobov ako tento interval dosiahnuť. Jedným z nich je použitie preddeličky (v anglickej literatúre sa používa termín prescaler).
\par
Preddelička je súčasťou obvodu časovača, ktorý nám umožňuje preskočiť periódy zdrojových hodín o mocninu 2, čím sa zníži frekvencia,
ale zároveň dostaneme dlhší rozsah časovača.
Hodnoty preddeličky bývajú pevne dané a často sú to hodnoty 1, 8, 64, 256 a 1024. Vzorec pre výpočet dĺžky cyklu spolu s preddeličkou je nasledovný \cite{atmelATmega64012801281}:

\begin{equation} \label{eq:cycle-length}
    \begin{aligned}
        Dĺžka\:cyklu & = \frac{1}{\frac{Frekvencia\:hodín}{Preddelička}} \\
                     & = \frac{Preddelička}{Frekvencia\:hodín}           \\
    \end{aligned}
\end{equation}

Pre časovač s frekvenciou 1 \acrshort{MHz} a hodnotami preddeličky 1, 8, 64, 256 s 1024 môžeme zostrojiť nasledujúcu tabuľku:
\begin{table}[!htbp]
    \begin{center}
        \begin{tabular}{|c|c|}
            \hline
            Hodnota preddeličky & Dĺžka cyklu časovača \\
            \hline
            1                   & 1 $\mu s$            \\
            8                   & 8 $\mu s$            \\
            64                  & 64 $\mu s$           \\
            256                 & 256 $\mu s$          \\
            1024                & 1024 $\mu s$         \\
            \hline
        \end{tabular}
        \caption{Dĺžka cyklu 1MHz časovača pri rôznych hodnotách preddeličky}
        \label{table:timerPrescaler}
    \end{center}
\end{table}

Pre dosiahnutie 1 sekundového intervalu potrebujeme poznať aj hodnotu registra,
pri ktorej tento interval uplynie. Tú dostaneme použitím nasledujúceho vzorca:

\begin{equation} \label{eq:register-value}
    \begin{aligned}
        Hodnota\:registra & = \frac{ \frac{1}{Cieľová\:frekvencia}} { \frac{Preddelička}{Frekvencia\:hodín}} - 1 \\
                          & = (\frac{Frekvencia\:hodín}{Preddelička \times Cieľová\:frekvencia}) - 1             \\
    \end{aligned}
\end{equation}

Časovače AVR aktualizujú svoju hodnotu iba pri každom tiknutí vstupných hodín.
Je teda potrebný tik, aby sa hodnota časovača dostala z maximálnej hodnoty späť na nulu. Preto v znázornenom vzorci
musíme znížiť počet tikov o jeden \cite{atmelATmega64012801281}. \par

Teraz sa vráťme späť k tomu, ako dosiahneme interval 1 sekundy.
Potrebujeme zistiť či existuje hodnota preddeličky, ktorá poskytne presné oneskorenie 1 \acrshort{Hz}. Jeden Hertz je rovný jednému cyklu za sekundu. Na výpočet použijeme vzorec
č.\ref{eq:register-value} a výsledky týchto výpočtov sú uvedené v tabuľku č. \ref{table:timerPrescalerValues}:

\begin{table}[!htbp]
    \begin{center}
        \begin{tabular}{|c|c|}
            \hline
            Hodnota preddeličky & Hodnota registra časovača \\
            \hline
            1                   & 999999                    \\
            8                   & 124999                    \\
            64                  & 15624                     \\
            256                 & 3905,25                   \\
            1024                & 975,5625                  \\
            \hline
        \end{tabular}
        \caption{Potrebná hodnota v registri 1MHz časovača pre dosiahnutie 1 sekundového intervalu}
        \label{table:timerPrescalerValues}
    \end{center}
\end{table}

Pri pohľade na tabuľku môžeme hneď usúdiť, že preddeličky s hodnotou 256 a 1024 nie sú vhodné pre dosiahnutie nášho cieľa. Hodnota v registri je vždy celé kladné číslo, a preto nemôžme tieto hodnoty použiť. Ďalším kritériom pre výber preddeličky je veľkosť
vypočítanej hodnoty. Náš časovač používa 16-bitový register, ktorého maximálna hodnota je $2^{16} -1$ (65365).
Ostáva nám teda len jedna možnosť preddeličky, a to 64. Preddeličky s hodnotou 1 a 8 vyžadujú hodnotu registra väčšiu ako je schopný uložiť.
Teda, pre dosiahnutie jedno sekundového intervalu pomocou 16 bitového časovača s frekvenciou 1MHz potrebujeme použiť preddeličku s hodnotou 64.
1 sekundu časovač dosiahne po napočítaní do 15624. \par
Čo však v prípade ak chceme dosiahnuť dlho trvajúce intervaly? Hodinu, deň, týždeň, mesiac prípadne rok?
V tomto prípade je možné implementovať preddeličku aj softvérovo. Pre jednoduchšie vysvetlenie si riešenie problému ukážeme na príklade. Uvažujme rovnaké parametre časovača: 1MHz frekvencia, 16 bitový register. Chceme dosiahnuť časový interval jednej hodiny. Pomocou vyššie popísanej techniky vytvoríme interval 1 sekundy.
Po každom uplynutí intervalu zvýšime hodnotu nášho počítadlá. 1 hodina je 3600 sekúnd. Teda vieme, že pokiaľ naše počítadlo dosiahne hodnotu 3600 ubehla 1 hodina.
Pseudokód danej implementácie by mohol vyzerať nasledovne:
\begin{lstlisting}[
    language=c
  ]  
Initialise counter to 0
WHILE forever
    IF timer value IS EQUAL TO 1 sec THEN
        Increment counter
        Reset timer
        IF counter value IS EQUAL TO 3600 THEN
            // 1 HOUR HAS PASSED
            Reset counter
        END IF
    END IF
END WHILE
\end{lstlisting}
Pri takejto implementácii je potrebné dať si pozor aj na najmenšie odchýlky od základného intervalu, ktorý je nastavený pomocou hardvérovej preddeličky.
Aj najmenšia odchýlka môže po sčítaní znamenať veľkú nepresnosť v požadovanom konečnom intervale \cite{cameraNewbieGuideAVR2015}.

\subsubsection{Časovače a prerušenia} \label{subsec:timers-interrupts}
\noindent \par
Pomocou AVR časovačov existujú dva spôsoby ako vykonať kód v presne danom čase. Prvým je použitie cyklu.
Pri tejto metóde dochádza k neustálej kontrole hodnoty registra časovača. Ak register dosiahne požadovanú hodnotu, vieme, že požadovaný interval skončil
a je možné vykonať inú akciu/akcie. Druhou možnosťou je presunúť zodpovednosť toho, kedy je akciu potrebné vykonať na hardvér AVR mikrokontroléra.
To je možné dosiahnuť pomocou prerušení časovača. AVR časovače podporujú niekoľko druhov prerušení.
Najčastejšie ide o Overflow (pretečenie) a Compare and Capture (porovnanie a zachytenie) prerušenia \cite{atmelATmega64012801281}. \par
Overflow prerušenie nastáva pri pretečení registra časovača. Časovač napočíta do svojej maximálnej hodnoty a v nasledujúcom cykle sa hodnota registra nastaví na 0.
Pri tomto pretečení časovač nastaví bit vo svojom stavovom registri a na základe toho je informovaná hlavná aplikácia o udalosti, ktorá nastala. Ak je dané prerušenie
povolené a zároveň sú globálne povolené všetky prerušenia, tak mikrokontrolér prestane vykonávať aktuálnu činnosť a spustí obsluhu prerušenia. Takýmto spôsobom
ponecháme rozhodnutie o vypršaní časového intervalu na hardvér AVR. Ak nechceme používať prerušenia, možnosťou je aj neustála softvérová kontrola bitu časovača v kontrolnom registri.
Práve na základe tohto bitu hardvér AVR vyvolá obsluhu prerušenia \cite{atmelATmega64012801281}. Nevýhodou je, že zatiaľ čo v tejto metóde je procesor neustále zamestnaný kontrolou
hodnoty registra, v prípade prerušenia môže namiesto toho vykonávať inú prospešnú činnosť. \par
Pre zostavenie rovnice na výpočet časového intervalu pretečenia môžeme použiť vzorec č.\ref{eq:cycle-length}. Ten hovorí o dĺžke jedného cyklu. Ak teda dĺžku cyklu
vynásobíme počtom možných hodnôt, dostaneme časový interval potrebný na pretečenie registra:

\begin{equation} \label{eq:overflow-frequency}
    \begin{aligned}
        Časový\:interval\:pretečenia =  2^{Veľkosť\:registra}  \times \frac{Preddelička}{Frekvencia\:hodín}
    \end{aligned}
\end{equation}

V nasledujúcej tabuľke č.\ref{table:overflow-frequency} vidíme porovnanie frekvencií a časových intervalov pretečenia registra časovača. Pre zostrojenie tabuľky sme uvažovali 16 bitový časovač
s frekvenciou 1MHz.  Vidíme, že čím väčšia hodnota preddeličky, tým menšia frekvencia
pretečenia registra. Z takmer nebadateľnej hodnoty pre ľudí, milisekunda, sme sa dostali až k intervalu viac ako 1 minúty.

\begin{table}[!htbp]
    \begin{center}
        \begin{tabular}{|c|c|c|}
            \hline
            Hodnota preddeličky & Frekvencia & Časový interval \\
            \hline
            1                   & 15,259 Hz  & 65,5 ms         \\
            8                   & 1,907 Hz   & 0,524 s         \\
            64                  & 0,2384 Hz  & 4,195 s         \\
            256                 & 0,0596 Hz  & 16,78 s         \\
            1024                & 0,0149 Hz  & 67,11 s         \\
            \hline
        \end{tabular}
        \caption{Frekvencie pretečenia registra pri rôznom nastavení preddeličky}
        \label{table:overflow-frequency}
    \end{center}
\end{table}

Použitie overflow prerušenia si ukážeme na už spomínanom príklade dosiahnutia 1 sekundového intervalu. V predchádzajúcich kapitolách sme si ukázali,
že pri použití 16-bitového časovača s frekvenciou 1MHz je možné dosiahnuť 1 sekundový interval po napočítaní do 15625 spolu s použitím preddeličky s hodnotou 64.
Zároveň vieme, že maximálna hodnota registra je 65535 a pri následnom pretečení nastane prerušenie. Vieme teda, že ak časovač dosiahne maximálnu hodnotu registra každú
sekundu, aj prerušenie bude vykonané pravidelne v tomto intervale. Z toho vyplýva, že časovač nesmie začať počítať od 0 ale od inej, presne danej hodnoty.
Pre matematické vyjadrenie počiatočnej hodnoty registra použijeme vzorec č.\ref{eq:register-value}. Pri vzorci sme uvažovali počiatočnú hodnotu 0. Tentokrát chceme
dosiahnuť to, aby po uplynutí intervalu časovač dosiahol svoju maximálnu hodnotu. Vzorec pre tento výpočet bude teda nasledovný:

\begin{equation} \label{eq:start-value}
    \begin{aligned}
        Počiatočná\:hodnota =  (2^{Veľkosť\:registra} - 1) - (\frac{Frekvencia\:hodín}{Preddelička \times Cieľová\:frekvencia})
    \end{aligned}
\end{equation}

Čo v našom konkrétnom prípade predstavuje:

\begin{equation}
    \begin{aligned}
        Počiatočná\:hodnota & =  (2^{Veľkosť\:registra} - 1) - 15625 \\
                            & =  (2^{16} - 1) - 15625                \\
                            & =  65535 - 15625                       \\
                            & =  49910
    \end{aligned}
\end{equation}

Hodnotu 49910 zapíšeme do kontrolného registra časovača na začiatku programu a následne pri
každej obsluhe overflow prerušenia. Pseudokód 1 sekundového intervalu s použitím overflow prerušenia vyzerá nasledovne:

\begin{lstlisting}[
    label={lst:oveflow-interrupt},
    language=c]  
Enable timer overflow interrupt
Enable global interrupts
Set timer prescaler to 64
Set timer register value to 49910

WHILE forever
END WHILE

ISR Timer Overflow
    Set timer register value to 49910
    // Do some work
END ISR

\end{lstlisting}

Spomínané Compare and Capture prerušenie sa vzťahuje hlavne k CTC módu časovača, preto si ho predstavíme až v kapitole č.\ref{subsec:ctc-mode}.

\subsubsection{CTC mód} \label{subsec:ctc-mode}
\noindent \par
Doteraz opisovaný spôsob fungovania AVR časovača sa vzťahuje na tzv. normálny  mód. V tomto móde časovač ráta od 0 po maximálnu hodnotu.
Je to základný a najjednoduchší spôsob fungovania AVR časovačov. Preto sme si na ňom predstavili spôsob akým tieto časovače fungujú a akým spôsobom sa dajú prakticky
používať. Komplikovanejším módom je tzv. „Clear timer on Compare“ (vynulovanie časovača pri porovnaní) mód, označovaný ako CTC mód. \par
CTC mód na hardvérovej úrovni porovnáva aktuálnu hodnotu časovača s požadovanou hodnotou. Ide o hardvérové porovnanie, ktoré je značne rýchlejšie ako porovnanie
softvérové. Ak časovač požadovanú hodnotu dosiahne, CTC mód nastaví bit v stavovom registri
a zároveň vynuluje hodnotu časovača. Týmto nám významným spôsobom uľahčuje prácu s časovačom. Nemusíme sa totiž starať o vynulovanie hodnoty časovača ako sme si ukázali
v kapitole č.\ref{subsec:timers-interrupts}. Zároveň nám ostáva možnosť použitia prerušenia na vykonanie akcie. Ak povolíme CTC prerušenia, tak po zapísaní bitu
do stavového registra AVR vykoná obsluhu prerušenia Compare and Capture. Aktuálna hodnota časovača je porovnávaná s hodnotou alebo hodnotami v tzv. Output Compare registrami.
16 bitové časovače majú tieto registre tri. Označované sú ako Output Compare Register A, Output Compare Register B a Output Compare Register C. My ich ďalej budeme nazývať
porovnávací register A, B, C. V 8 bitových časovačoch nájdeme len prvé dva. Vďaka viacerým porovnávacím registrom, dokážeme jeden časovač využiť na generovanie viacerých
časových intervalov \cite{atmelATmega64012801281}. \par
Po každom tiku časovača je aktuálna hodnota  porovnaná s hodnotami v porovnávacích registroch. Každý porovnávací register funguje
samostatne a podporuje nezávislé generovanie prerušení. Je dôležité poznamenať, že hodnota časovača je vynulovaná len v prípade zhody hodnoty s porovnávacím registrom A.
To je obzvlášť dôležité ak chceme používať jeden časovač na generovanie viacerých intervalov. Ak by sme do porovnávacieho registra B zapísali hodnotu väčšiu ako
do porovnávacieho registra A, časovač nikdy nedosiahne hodnotu v registri B. Rovnako to platí aj pre register C. Z uvedeného taktiež vyplýva, že maximálna hodnota
časovača už nie je kapacita jeho registra, ale hodnota, ktorá je zapísaná v porovnávacom registri A \cite{atmelATmega64012801281}. \par
Ako príklad znovu využijeme 1 sekundový interval, označíme ho ako interval 1.
Pridáme k nemu ešte interval 200 milisekúnd, ktorý chceme, aby odštartoval v rovnaký moment ako 1 sekundového interval. Pomenujeme ho ako interval 2. Pre jednoduchšiu
predstavu je na obrázku č.\ref{figure:interval-timeline} zobrazená časová os spolu s označenými intervalmi.


\begin{figure}[!h]
    \centering
    \includegraphics[width=0.85\textwidth]{img/interval-timeline.drawio.png}
    \caption{Časová os požadovaných intervalov}
    \label{figure:interval-timeline}
\end{figure}

Jeden časovač chceme použiť na dva rôzne intervaly. Preto v CTC móde potrebujeme 2 porovnávacie registre. Použijeme register A a porovnávací register B. Znovu uvažujeme
16 bitový časovač s frekvenciou 1MHz. V predchádzajúcich kapitolách sme vypočítali, že pri použití preddeličky s hodnotou 64, časovač dosiahne 1 sekundový interval
po napočítaní do 15624. Potrebujeme ešte vypočítať porovnávaciu hodnotu pre interval 200 milisekúnd. Použijeme na to vzorec č.\ref{eq:register-value}.

\begin{equation}
    \begin{aligned}
        Hodnota\:registra & = (\frac{Frekvencia\:hodín}{Preddelička \times Cieľová\:frekvencia}) - 1 \\
                          & = (\frac{1 \times 10^{6}}{64 \times 5}) - 1                              \\
                          & = (\frac{1 \times 10^{6}}{320}) - 1                                      \\
                          & = 3125 - 1                                                               \\
                          & = 3124                                                                   \\
    \end{aligned}
\end{equation}

Vypočítali sme obidve potrebné porovnávacie hodnoty. Porovnávacia hodnota intervalu 1 je väčšia ako hodnota pre interval 2. Väčšiu hodnotu teda zapíšeme do registra A a
menšiu do registra B. Časovač zabezpečí neustále porovnávanie týchto hodnôt s aktuálnou hodnotou časovača. Ak zároveň povolíme Compare and Capture prerušenia pre register
A a aj register B, mikrokontrolér vygeneruje prerušenia pri zhode hodnôt. Vďaka tomu dokážeme spúšťať kód v presných časových intervaloch. Zmena hodnoty časovača spolu
s generovaním prerušení je zobrazené na obrázku č.\ref{figure:ctc-timer-value}.
\begin{figure}[!h]
    \centering
    \includegraphics[width=0.85\textwidth]{img/ctc-timer-value.png}
    \caption{Zmena hodnoty časovača v CTC móde}
    \label{figure:ctc-timer-value}
\end{figure}

Pseudokód daného riešenia vyzerá nasledovne:

\begin{lstlisting}[
    label={lst:ctc-interrupt},
    language=c]  

Enable timer Compare nad Capture A interrupt
Enable timer Compare nad Capture B interrupt
Enable global interrupts

Set Output Compare Register A value to 15624
Set Output Compare Register B value to 3124

Set timer mode to CTC
Set timer prescaler to 64

WHILE forever
END WHILE

ISR Compare nad Capture A
    // Do some work at 1 second interval
END ISR

ISR Compare nad Capture B
    // Do some work at 200 milliseconds interval 
END ISR

\end{lstlisting}

Poznamenať by sme mali, že pri použití CTC módu nie je možné nastaviť časovač tak, aby generoval akékoľvek dva, resp. tri intervaly. Pre časovač je možné nastaviť
len jednu preddeličku. Z toho vyplýva, že aj všetky intervaly musia byť kompatibilné s danou preddeličkou. Ak náš predchádzajúcu príklad trochu upravíme
a 200 milisekundový interval nahradíme 250 milisekundovým dostaneme spomínané nekompatibilné hodnoty. Pomocou vzorca \ref{eq:register-value} sa pokúsime vypočítať
porovnávaciu hodnotu:
\begin{equation}
    \begin{aligned}
        Hodnota\:registra & = (\frac{Frekvencia\:hodín}{Preddelička \times Cieľová\:frekvencia}) - 1 \\
                          & = (\frac{1 \times 10^{6}}{64 \times 4}) - 1                              \\
                          & = (\frac{1 \times 10^{6}}{256}) - 1                                      \\
                          & = 3906,25 - 1                                                            \\
                          & = 3905,25                                                                \\
    \end{aligned}
\end{equation}

Vidíme, že výsledkom nie je celé číslo. Teoretickým riešením by mohlo byť zaokrúhlenie danej hodnoty na celé číslo. Tým by sme však do celého procesu vniesli nepresnosť,
ktorá je vo väčšine prípadov neakceptovateľná.

\subsubsection{PWM mód}
\noindent \par
Pulse Width Modulation (impulzová šírková modulácia) alebo aj PWM je technika, pomocou ktorej je možné ovládať analógové zariadenia pomocou digitálneho signálu.
Jednoduchým príkladom je stmievanie led žiarovky alebo ovládanie rýchlosti motora. PWM signál nie je reálnym analógovým signálom, len ho určitým spôsobom dokáže napodobniť.
Pomocou digitálneho ovládania dokážeme rýchlo napájanie vypínať alebo zapínať a riadiť tak prúd dodávaný do zoriadenia. Ak necháme prúd zapnutý dlhší čas ako čas, v ktorom
je vypnutý, tak zvýšime priemernú úroveň výkonu. V opačnom prípade priemernú úroveň výkonu znížime. Trvanie „času zapnutia“ sa nazýva šírka impulzu.
Ak chceme získať rôzne analógové hodnoty, zmeníme šírku impulzu. Na vyjadrenie dĺžky času zapnutia sa používa termín Duty cycle (pracovný cyklus). Udáva sa v percentách
a hovorí o dĺžke „času zapnutia“  počas jedného intervalu \cite{WhatPWMSignal}.
\begin{figure}[!h]
    \centering
    \includegraphics[width=0.85\textwidth]{img/duty-cycle.jpg}
    \caption{Ukážka rôznej hodnoty striedy v PWM signále}
    \source{\cite{WhatPWMSignal}}
    \label{figure:pwm-signal}
\end{figure}

Ako sme si ukázali, pre generovanie PWM signálu potrebujeme neustále v určitých časových intervalov „vypínať“ a „zapínať“ prúd. Na to môžu výborne poslúžiť časovače.
Tie dokážu na presne dané piny mikrokontroléra zapisovať logickú jednotku, resp. nulu čoho výsledkom môže byť PWM signál. To je možné docieliť
pomocou špeciálnych PWM módov, ktoré si predstavíme v ďalšej časti diplomovej práce.

\noindent \par
Fast PWM mód je jedným z podporovaných modov v AVR časovačoch. V tomto móde sa časovač jednoducho počíta od 0 po maximálnu kapacitu registra. Pri počítaní porovnáva svoju hodnotu, s hodnotami v porovnávacích registroch.
Používajú sa rovnaké porovnávacie registre ako v už spomínanom CTC móde. Tieto registre sú naviac namapované na externé piny mikrokontroléra.
Na výstupné piny je zapísaná logická jednotka, keď je časovač na 0 a logická nula, keď sa časovač zhoduje s výstupným porovnávacím registrom. Čím vyššia je hodnota
v porovnávacom registri, tým je na pine nastavená väčšia strieda \cite{shirriffSecretsArduinoPWM}. \par
Existujú aj rôzne variácie Fast PWM módu, kedy je možné nastaviť maximálnu hodnotu časovača a tým meniť frekvenciu. \par
Na obrázku č.\ref{figure:fast-pwm-mode} môžeme vidieť zmenu hodnoty v 8 bitovom AVR časovači spolu s PWM signálom, ktorý je schopný vygenerovať.
\begin{figure}[!h]
    \centering
    \includegraphics[width=0.85\textwidth]{img/fast-pwm-graph.png}
    \caption{Fast PWM mód použitý v 8 bitovom časovači}
    \source{\cite{shirriffSecretsArduinoPWM}}
    \label{figure:fast-pwm-mode}
\end{figure}

Porovnávací register A je vyznačený zelenou farbou, zatiaľ čo register B je označený farbou červenou. Obrázok č.\ref{figure:fast-pwm-mode} znázorňuje následné hodnoty:
\begin{itemize}
    \item Porovnávací register A
          \subitem{Hodnota porovnávacieho registra A: 180}
          \subitem{Strieda pre porovnávací register A: (180 + 1)/256 = 70,7\%}
    \item Porovnávací register B
          \subitem{Hodnota porovnávacieho registra B: 50}
          \subitem{Strieda pre porovnávací register B: (50 + 1)/256 = 19,9\%}
\end{itemize}

Pri výpočte striedy si možeme všimnúť, že k hodnote v registri sme pripočítali 1. Je to kvôli tomu, že AVR časovaču trvá jednu periódu navyše pokiaľ hodnotu
na danom pine zmení \cite{shirriffSecretsArduinoPWM}.

\noindent \par
Ďalším módom je tzv. Phase-Correct PWM mód. V tomto móde časovač počíta od 0 po maximálnu hodnotu registra a následne sa hodnota nevynuluje
ale časovač znovu počíta od najväčšej hodnoty po 0. Na výstupné piny je zapísaná logická jednotka, keď je časovač na 0. Pri počítaní smerom nahor sa pri dosiahnutí porovnávacieho registra výstupný pin vypne
a následne pri počítaní smerom nadol zapne. Výsledkom je symetrický výstup \cite{shirriffSecretsArduinoPWM}. Na obrázku  č.\ref{figure:phase-correct-pwm-mode} si môžeme všimnúť,
že dôsledkom počítania aj smerom nadol je polovičná frekvencia oproti režimu Fast PWM.

\begin{figure}[!h]
    \centering
    \includegraphics[width=0.85\textwidth]{img/phase-correct.png}
    \caption{Phase-Correct PWM mód použitý v 8 bitovom časovači}
    \source{\cite{shirriffSecretsArduinoPWM}}
    \label{figure:phase-correct-pwm-mode}
\end{figure}

Porovnávací register A je vyznačený zelenou farbou, zatiaľ čo register B je označený farbou červenou. Obrázok č.\ref{figure:phase-correct-pwm-mode} znázorňuje následné hodnoty:
\begin{itemize}
    \item Porovnávací register A
          \subitem{Hodnota porovnávacieho registra A: 180}
          \subitem{Strieda pinu porovnávaceho registra A: (180 + 1)/256 = 70,7\%}
    \item Porovnávací register B
          \subitem{Hodnota porovnávacieho registra B: 50}
          \subitem{Strieda pinu porovnávaceho registra B: (50 + 1)/256 = 19,9\%}
\end{itemize}