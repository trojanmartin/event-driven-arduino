
V diplomovej práci sme sa venovali problematike udalosťami riadeného programovania s použitím prerušení mikrokontroléra. Na úvod sme si prestavili pojmy ako mikrokontrolér
a prerušenie. Neskôr sme opísali spôsob fungovania AVR časovačov, ich možnú konfiguráciu a spôsob použitia v programe. Pokračovali sme predstavením programovacích paradigiem a modelovaním
udalosťami riadeného programovania pomocou stavových automatov. \par
V praktickej časti práce sme vyvinuli softvérovú knižnicu, ktorej úlohou je modelovanie programu pomocou stavových automatov s použitím prerušení mikrokontroléra Arduino
Mega 2560. Prerušenia sú použité na generovanie udalostí, ktorých dôsledkom je zmena stavu stavového automatu. Knižnica umožňuje jednoduchú konfiguráciu a použitie
prerušení aj pre používateľov, ktorí nemajú hĺbkové vedomosti prerušovacieho systému mikrokontroléra. Implementované komponenty podporujú konfiguráciu
časovačov 1, 3, 4 a 5, externých prerušení 0 až 7 a prerušení o zmene hodnoty pinu 0 až 2. Vytvorené riešenie rovnako ponúka aj možnosť generovania
udalostí softvérovým spôsobom bez použitia prerušení. Pre demonštráciu riešenie obsahuje implementáciu softvérového generátora udalostí, ktorý sleduje stav pripojeného
tlačidla.
\par
Vytvorená knižnica umožňuje program namodelovať pomocou stavových automatov. Výsledkom toho je lepšia čitateľnosť kódu a nižšia náročnosť zmeny existujúcich programov.
Vďaka jednoduchej a intuitívnej konfigurácii prerušení je použiteľná aj pre menej zdatných používateľov. Pokročilým používateľom je umožnené skombinovanie použitia
knižnice s vlastnou obsluhou prerušení. Medzi nevýhody použitia knižnice je jej nekompatibilita s inými mikrokontrolérmi Arduino. Vývoj aplikácie prebiehal na platforme
Arduino Mega 2560 a konfigurácia prerušení je priamo naviazaná na použitý procesor mikrokontroléra. Medzi ďalšie nevýhody patrí nižšia flexibilita konfigurácie
prerušení oproti manuálnej konfigurácii. \par
V ďalšom vývoji knižnice by sme sa mohli zamerať na použiteľnosť knižnice aj na iných typoch mikrokontrolérov Arduino. Pomocou podmienenej kompilácie kódu
by sme vedeli dosiahnuť kompatibilitu s ďalšími verziami mikrokontrolérov AVR. Rozšírením podpory aj pre iné prerušenia by sme zasa dosiahli väčšiu flexibilitu
riešenia.