\section{Arduino}
\noindent \par
V súčasnej dobe patrí platforma Arduino medzi populárne elektronické platformy na svete.
Jeho výhodou je otvorený zdrojový kód, jednoduchosť použitia, ale aj možnosť vytvárať komplexnejšie projekty.
Medzi základné vlastnosti mikrokontrolérov Arduino patrí:

\begin{itemize}
    \item Cena: Arduino je pomerne lacné v porovnaní s ostatnými mikrokontrolérmi. Jeho maximálna cena dosahuje menej ako 50 eur.
    \item Podpora pre multiplatformový vývoj: k Arduinu je poskytované aj integrované vývojové prostredie. To je možné spustiť na operačných systémoch Windows, Linux
          ale aj Macintosh OSX.
    \item Rozšíriteľnosť platformy Arduino: Platforma nie je limitovaná na používanie len preddefinovaných knižníc a nástrojov. Je možné ju rozširovať o nové komponenty. Vďaka tomu, že Arduino používa AVR mikročipy je možné aj použitie nástrojov a kódov špecializovaných pre AVR.
\end{itemize}

Vďaka týmto spomenutým aspektom je Arduino využívané úplnými začiatočníkmi, ale aj prokročilejšími používateľmi \cite{WhatArduinoArduino}.